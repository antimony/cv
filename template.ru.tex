%% start of file `template.tex'.
%% Copyright 2006-2015 Xavier Danaux (xdanaux@gmail.com).
%
% This work may be distributed and/or modified under the
% conditions of the LaTeX Project Public License version 1.3c,
% available at http://www.latex-project.org/lppl/.


\documentclass[11pt,a4paper,sans]{moderncv}        % possible options include font size ('10pt', '11pt' and '12pt'), paper size ('a4paper', 'letterpaper', 'a5paper', 'legalpaper', 'executivepaper' and 'landscape') and font family ('sans' and 'roman')

% moderncv themes
\moderncvstyle{casual}                             % style options are 'casual' (default), 'classic', 'banking', 'oldstyle' and 'fancy'
\moderncvcolor{blue}                               % color options 'black', 'blue' (default), 'burgundy', 'green', 'grey', 'orange', 'purple' and 'red'
%\renewcommand{\familydefault}{\sfdefault}         % to set the default font; use '\sfdefault' for the default sans serif font, '\rmdefault' for the default roman one, or any tex font name
%\nopagenumbers{}                                  % uncomment to suppress automatic page numbering for CVs longer than one page

% character encoding
%\usepackage[utf8]{inputenc}                       % if you are not using xelatex ou lualatex, replace by the encoding you are using
%\usepackage{CJKutf8}                              % if you need to use CJK to typeset your resume in Chinese, Japanese or Korean
\usepackage[utf8x]{inputenc} % Включаем поддержку UTF8
\usepackage[T2A]{fontenc}    
\usepackage[english,main=russian]{babel}
% adjust the page margins
\usepackage[scale=0.75]{geometry}
%\setlength{\hintscolumnwidth}{3cm}                % if you want to change the width of the column with the dates
%\setlength{\makecvheadnamewidth}{10cm}            % for the 'classic' style, if you want to force the width allocated to your name and avoid line breaks. be careful though, the length is normally calculated to avoid any overlap with your personal info; use this at your own typographical risks...

% personal data
\name{Михаил}Сурин}
\title{Разработчик программного обеспечения}                               % optional, remove / comment the line if not wanted
%\address{street and number}{postcode city}{country}% optional, remove / comment the line if not wanted; the "postcode city" and "country" arguments can be omitted or provided empty
\phone[mobile]{+7~(926)~731~0019}                   % optional, remove / comment the line if not wanted; the optional "type" of the phone can be "mobile" (default), "fixed" or "fax"
%\phone[fixed]{+2~(345)~678~901}
\email{to@mhl.su}                               % optional, remove / comment the line if not wanted
%\homepage{www.johndoe.com}                         % optional, remove / comment the line if not wanted
\social[linkedin]{msurin}                        % optional, remove / comment the line if not wanted
\social[twitter]{mhlsu}                             % optional, remove / comment the line if not wanted
\social[github]{antimony}                              % optional, remove / comment the line if not wanted
\social[facebook]{mikhail.surin}                              % optional, remove / comment the line if not wanted
\social[skype]{deus_indeed}                              % optional, remove / comment the line if not wanted
%\extrainfo{additional information}                 % optional, remove / comment the line if not wanted
\photo[64pt][0.4pt]{picture}                       % optional, remove / comment the line if not wanted; '64pt' is the height the picture must be resized to, 0.4pt is the thickness of the frame around it (put it to 0pt for no frame) and 'picture' is the name of the picture file
%\quote{Some quote}                                 % optional, remove / comment the line if not wanted

% bibliography adjustements (only useful if you make citations in your resume, or print a list of publications using BibTeX)
%   to show numerical labels in the bibliography (default is to show no labels)
\makeatletter\renewcommand*{\bibliographyitemlabel}{\@biblabel{\arabic{enumiv}}}\makeatother
%   to redefine the bibliography heading string ("Publications")
%\renewcommand{\refname}{Articles}

% bibliography with mutiple entries
%\usepackage{multibib}
%\newcites{book,misc}{{Books},{Others}}
%----------------------------------------------------------------------------------
%            content
%----------------------------------------------------------------------------------
\begin{document}
%\begin{CJK*}{UTF8}{gbsn}                          % to typeset your resume in Chinese using CJK
%-----       resume       ---------------------------------------------------------
\makecvtitle

\section{Образование}
\cventry{2002--2008}{Теоретическая физика и прикладная математика}{Уральский Государственный Технический Университет-УПИ}{Екатеринбург}{\textit{Инженер-физик}}{}  % arguments 3 to 6 can be left empty

\section{Дипломная работа}
\cvitem{название}{\emph{Исследование электронной структуры и магнитного состояния актиноидных металлов середины 5f ряда }}
\cvitem{рецензенты}{A.O. Шориков, Л.Ф. Гаврилов, Ю.В. Егоров}
\cvitem{описание}{Вычисление электронной структуры и магнитного состояния актиноидных металлов середины 5f ряда с использованием метода LDA+U+SO}

\section{Опыт работы}
\cventry{2016--}{Руководитель группы разработки бэкенда}{Яндекс}{Москва}{}{Яндекс.Афиша\newline{} Python, Flask, Celery, MongoDB}%
Достижения:%
\begin{itemize}%
\item Глобальный рефакторинг архитектурных проблем;%
\item Частичный перевод на микросервисную архитектуру;%
\item Ускорение работы фоновых задач с 4 часов до 15 минут;%
\item Ускорение работы API до 10 раз;%
\end{itemize}%
\cventry{2014--2016}{Старший разработчик ПО}{Яндекс}{Москва}{}{Яндекс.Билеты\newline{} Java, Spring MVC, MongoDB}%
Достижения:%
\begin{itemize}%
\item Разработка системы создания схем планов залов для дизайнеров;%
\end{itemize}%
\cventry{2013--2014}{Разрабочик ПО}{Яндекс}{Москва}{}{Яндекс.Билеты\newline{} Java, Spring MVC, MongoDB}%
Достижения:%
\begin{itemize}%
\item Разработка архитектуры приложения:%
 \begin{itemize}%
  \item Выбор БД;%
  \item Разработка структуры API;%
  \item Разработка панели администратора;%
  \end{itemize}%
\item Интеграция с продавцами билетов%
\end{itemize}
\cventry{2012--2013}{Разработчик ПО}{Рамблер}{Москва}{}{Рамблер-Касса\newline{}ASP.NET MVC, MSSQL, javascript}
\cventry{2012}{Developer}{McLeod Software}{Екатеринбург}{}{Разработка приложения по управлению грузоперевозками. http://www.mcleodsoftware.com/\newline{}ASP.NET, VB, javacript, MSSQL}
\cventry{2011-2012}{Team Lead}{WebTeam}{Екатеринбург}{}{ASP.net based application development\newline{}ASP.NET MVC, NHibernate, MSSQL, MongoDB}
Достижения:%
\begin{itemize}%
\item Разработка веб-приложения:%
 \begin{itemize}%
  \item База данных;%
  \item Кэширование;%
  \item Вебсайт;%
  \item Панель администрирования;%
  \end{itemize}%
\item Интеграция с Facebook graph%
 \begin{itemize}%
  \item Получение списка друзей;%
  \item Построение графа отношения;%
  \item Вычисление расстояния пользователя от человека работающего в желаемой компании;%
 \end{itemize}
\end{itemize}
\cventry{2010-2012}{Веб-разработчик}{Яксон}{Ekaterinburg}{}{Кроссдоменная доска объявлений о вакансиях\newline{}ASP.NET, NHibernate, Lucene, Solr, javascript}
Достижения:%
\begin{itemize}%
\item Единый портал кросс-доменной авторизации%
\item Полнотекстовый поиск на SOLR%
\item Миграция с ASP.net WebPages на ASP.net MVC%
\item Встраиваемые виджеты карточек вакансий для сторонних сайтов%
\end{itemize}
\cventry{2008-2010}{Разработчик}{eConsortian}{Екатеринбург}{}{Разработка проектов на ASP.net\newline{} .NET, MSSQL, javascript, ActionScript, Wowza}
Проекты:%
\begin{itemize}%
\item ArtBanc. www.arcbanc.com%
\item StoryPalz. www.storypalz.com%
\item PointHR. www.pointhr.com%
\item DeliveryMaxx. www.deliverymaxx.com%
\end{itemize}
\cventry{2006-2010}{Системный администратор}{Уральский завод энергетических машин}{Екатеринбург}{}{Поддержка сети и разработка системы управления рабочим временем для газотурбинного завода\newline{}}

\section{Языки}
\cvitemwithcomment{Английский}{Upper-intermediate}{}
\cvitemwithcomment{Французский}{Базовые знания}{}

\section{Языки программирования}
\cvitemwithcomment{Java}{Основной язык программирования}{5 лет}{}
\cvitemwithcomment{C\#}{Основной язык программирования}{10 лет}{}
\cvitemwithcomment{Python}{Основной язык программирования}{2 года}
\cvitemwithcomment{Javascript}{Средний уровень}{10 years}{}
\cvitemwithcomment{CSS}{Средний уровень}{10 years}{}
\cvitemwithcomment{Actionscript 3.0}{Средний уровень}{3 years}{}
\cvitemwithcomment{PHP}{Базовый уровень}{}
\cvitemwithcomment{Ruby}{Базовый уровень}{}

\section{Фреймворки и Технологии}
\cvdoubleitem{Java}{Java 8/9, Spring, JAX-WS, Jetty, Jersey, Grizzly, Guice, log4j, Hibernate}{Python}{Django, Flask, Celery}
\cvdoubleitem{C\#}{.NET, NHibernate, Entity Framework}{Ruby}{Ruby on rails}
\cvdoubleitem{Javascript}{jQuery, Bootstrap, node.js}{PHP}{Symphony, Yii}
\cvdoubleitem{CSS}{LESS, SASS}{SQL}{MSSQL, PostgreSQL}
\cvdoubleitem{NoSQL}{MongoDB, Riak, Redis, Memcached}{Other}{Docker, Gradle, Ant, Amazon Web Services}

\section{Увлечения}
\cvitem{Волейбол}{любительский уровень}
\cvitem{Интеллектуальный игры}{Что? Где? Когда?, Брейн-Ринг, Своя игра}
\cvitem{Музыка}{Классический рок-н-ролл, 60е, 70е}

% Publications from a BibTeX file without multibib
%  for numerical labels: \renewcommand{\bibliographyitemlabel}{\@biblabel{\arabic{enumiv}}}% CONSIDER MERGING WITH PREAMBLE PART
%  to redefine the heading string ("Publications"): \renewcommand{\refname}{Articles}

% Publications from a BibTeX file using the multibib package
%\section{Publications}
%\nocitebook{book1,book2}
%\bibliographystylebook{plain}
%\bibliographybook{publications}                   % 'publications' is the name of a BibTeX file
%\nocitemisc{misc1,misc2,misc3}
%\bibliographystylemisc{plain}
%\bibliographymisc{publications}                   % 'publications' is the name of a BibTeX file

\clearpage
%-----       letter       ---------------------------------------------------------
% recipient data
\recipient{Company Recruitment team}{Company, Inc.\\123 somestreet\\some city}
\date{January 01, 1984}
\opening{Dear Sir or Madam,}
\closing{Yours faithfully,}
\enclosure[Attached]{curriculum vit\ae{}}          % use an optional argument to use a string other than "Enclosure", or redefine \enclname
\makelettertitle

Lorem ipsum dolor sit amet, consectetur adipiscing elit. Duis ullamcorper neque sit amet lectus facilisis sed luctus nisl iaculis. Vivamus at neque arcu, sed tempor quam. Curabitur pharetra tincidunt tincidunt. Morbi volutpat feugiat mauris, quis tempor neque vehicula volutpat. Duis tristique justo vel massa fermentum accumsan. Mauris ante elit, feugiat vestibulum tempor eget, eleifend ac ipsum. Donec scelerisque lobortis ipsum eu vestibulum. Pellentesque vel massa at felis accumsan rhoncus.

Suspendisse commodo, massa eu congue tincidunt, elit mauris pellentesque orci, cursus tempor odio nisl euismod augue. Aliquam adipiscing nibh ut odio sodales et pulvinar tortor laoreet. Mauris a accumsan ligula. Class aptent taciti sociosqu ad litora torquent per conubia nostra, per inceptos himenaeos. Suspendisse vulputate sem vehicula ipsum varius nec tempus dui dapibus. Phasellus et est urna, ut auctor erat. Sed tincidunt odio id odio aliquam mattis. Donec sapien nulla, feugiat eget adipiscing sit amet, lacinia ut dolor. Phasellus tincidunt, leo a fringilla consectetur, felis diam aliquam urna, vitae aliquet lectus orci nec velit. Vivamus dapibus varius blandit.

Duis sit amet magna ante, at sodales diam. Aenean consectetur porta risus et sagittis. Ut interdum, enim varius pellentesque tincidunt, magna libero sodales tortor, ut fermentum nunc metus a ante. Vivamus odio leo, tincidunt eu luctus ut, sollicitudin sit amet metus. Nunc sed orci lectus. Ut sodales magna sed velit volutpat sit amet pulvinar diam venenatis.

Albert Einstein discovered that $e=mc^2$ in 1905.

\[ e=\lim_{n \to \infty} \left(1+\frac{1}{n}\right)^n \]

\makeletterclosing

%\clearpage\end{CJK*}                              % if you are typesetting your resume in Chinese using CJK; the \clearpage is required for fancyhdr to work correctly with CJK, though it kills the page numbering by making \lastpage undefined
\end{document}


%% end of file `template.tex'.
